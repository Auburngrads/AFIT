%Ignored glossary entries
\newglossaryentry{QSCM}
{
    name={QS Computational Module},
    type={ignored},
    user1={QS Computational Module},
    description={no good description}
}

\newglossaryentry{am}
{
    name={Autonomous Module},
    type={ignored},
    user1={Autonomous Module},
    description={no good description}
}

\newglossaryentry{alm}
{
    name={Algorithmic Module},
    type={ignored},
    user1={Algorithmic Module},
    description={no good description}
}
%%%%%%%%%%%%%%%%%%%%%%%%%%%%%%%%%%%%%%%%%%%%%%%%%%%%%%%%%%%%%%%%%%%%%%%%%%%%%%%%%%%%
\newglossaryentry{confabulated}
{
  name=confabulated,
  description={``Psychiatry: Fabricate imaginary experiences as compensation for loss of memory \citep{Oxford2014} (URL Date 17 May 2014)''}
}

\newglossaryentry{Type I}
{
  name=Type I,
  description={Refers to autonomous cognitive processing}
}

\newglossaryentry{Type II}
{
  name=Type II,
  description={Refers to reflective cognitive processing}
}


\newglossaryentry{autonomous}
{
  name=autonomous,
  description={Refers to cognitive \gls{Type I} processing}
}

\newglossaryentry{reflective}
{
  name=reflective,
  description={The conscious, \emph{reflective mind}, is explicit, effortful, pattern-completion (i.e., hypothetical) decision making which supports slow deliberation, and uses \gls[format=hyperbf]{WM} \citep{anderson2005cognitive, kahneman2011thinking, Stanovich2013DualProcess}.}
}

\newglossaryentry{algorithmic}
{
  name=algorithmic,
  description={The conscious, \emph{algorithmic mind}, is is responsible for sequencing behavior and \gls{cognitive decoupling} \citep{Stanovich2013DualProcess}.}
}

\newglossaryentry{mental simulation}
{
  name=mental simulation,
  description={``The mental processes by which people construct scenarios, or examples, resemble the running of the simulation model. Mental simulation appears to be used to make predictions, assess probabilities and evaluate casual statements \citep{kahneman1981simulation}''},
  user1={Mental Simulation}
}

\newglossaryentry{abductive}
{
  name=abductive,
  description={(reasoning) An inference to the best explanation \citep{henson2012semantic}. Abduction is inferring a case (particular abstract relationship) from a rule (abstract, general claim) and a result (empirical observation) \citep{shanahan1996robotics}.
  Generating a series of competing plausible explanatory hypothesis, and choosing the best based on some set of criteria \citep{henson2012semantic}}
}
\newglossaryentry{inductive}
{
  name=inductive,
  description={(reasoning) A generalization is inferred given a specific case, i.e., from the specific to the general.
In induction the conclusion is not assured to be correct \citep{svennevig2001abduction}}
}
\newglossaryentry{deductive}
{
  name=deductive,
  description={(reasoning) A general rule is applied to a specific case, i.e., from the general to the specific.
Deductive reasoning is the only way to achieve a provable, logical, solution \citep{svennevig2001abduction}}
}
\newglossaryentry{transductive}
{
  name=transductive,
  description={(reasoning) Transductive reasoning is inferring from one specific experience to another specific case \citep{vapnik2006empirical}}
}
% source-(Butt, 2013
\newglossaryentry{MAS}
{
  type=\acronymtype,
  name=MAS,
  description={Multi\-Agent System (MAS). ``A multi-agent system (i.e., a society of \glspl{agent}) is a community of autonomous entities each of which perceives, decides, and acts on its own, in accordance with its own interest, but may also cooperate with others to achieve common goals and objectives \citep{sun2006cognition}'' MASa are primarily used to analyze socio-culteral aspect of cognition \citep{butt2013soar}},
  user1={Multi-Agent System (MAS)},
  first={Multi-Agent System (MAS)},
  plural={MASs}
}

% source-(Butt, 2013
\newglossaryentry{agent}
{
  name=agent,
  description={An agent is a physical or virtual entity, with some capability of acting or reasoning \citep{ferber1999multi}},
  plural={agents}
}

% source-(Butt, 2013
\newglossaryentry{cognitive architecture}
{
  name=cognitive architecture,
  description={A cognitive architecture is both a theory and the representation of that theory in a computer-based modeling tool, referred to as a \glsreset{CMA}\gls{CMA}. The theory is ``The fixed (or slowly varying) structure that forms the framework for the immediate processes of cognitive performance and learning \citep{newell1990unified}.'' No \gls{CMA} claims to fully model cognitive processes},
  plural=cognitive architectures,
  user1={Cognitive Architectures}
}
\newglossaryentry{cognitive framework}
{
  name=cognitive framework,
  description={The theoretical portion-only of a cognitive architecture, or a cognitive architecture with no computational implementation},
  plural=cognitive frameworks
}
%source-(Stanovich)
\newglossaryentry{dysrationalia}
{
  name=dysrationalia,
  description={``The inability to think and behave rationally despite having adequate intelligence. An analogue of the word \textbf{dyslexia}.  \citep{Stanovich2009IntTest}''}
}

\newglossaryentry{chunk}
{
  name=chunk,
  description={A single declarative unit of knowledge. In some cognitive theories, including \gls{ACTR}, elements of declarative knowledge are called chunks \citep{anderson2004integrated}},
  plural=chunks
}

\newglossaryentry{chunk type}
{
  name=chunk type,
  description={A category of a chunk, a declarative unit of knowledge (e.g., bird, dog or house) \citep{anderson2004integrated}}
}

\newglossaryentry{slot}
{
  name=slot,
  description={(In \gls{ACTR}) A chunk category attribute (e.g., color or size) \citep{anderson2004integrated}.  Comparable to a field in a database table, but there is no relational functionality},
  plural=slots
}

\newglossaryentry{autonomous learning}
{
  name=autonomous learning,
  description={\Glspl{cognitive architecture} should include only minimal initial structures and independently learn from their own experiences\citep{sun2004desiderata}},
  user1={Autonomous Learning}
}

\newglossaryentry{DM}
{
  type=\acronymtype,
  name=DM,
  description={Declarative Memory (DM). Also called declarative knowledge, corresponds to things we are aware we know and can usually describe to others. \gls{DM} is generally used to represent episodic memory and \gls{semantic memory}  \citep{Anderson2007HowCan}},
  user1={Declarative Memory (DM)},
  first={declarative memory (DM)}
}
\newglossaryentry{PSSH}
{
  type=\acronymtype,
  name=PSSH,
  description={physical symbol system hypothesis (PSSH) \citep{newell1961computer}},
  first={physical symbol system hypothesis (PSSH)}
}

\newglossaryentry{LTM}
{
  type=\acronymtype,
  name=LTM,
  description={Long-term Memory (LTM). Relatively permanent storage of memory. ''Information must be rehearsed before it can be moved into a relatively permanent Long-term memory \citep[176]{anderson2005cognitive} ''},
  user1={Long-term Memory},
  first={Long-term Memory (LTM)},
  plural={LTMs}
}

\newglossaryentry{proposition}
{
    name=proposition,
    description={``Cognitive Science: The smallest unit of knowledge that can stand as a separate assertion; that is, the smallest unit one can meaningfully judge as true or false \citep[147]{anderson2005cognitive},
    \newline
    \newline
    Set Theory: A proposition, $P$, is a declaration that can be either true or false, but not both \citep{johnson2013hypernetworks}''},
    user1={Proposition},
    plural={propositions}
}

\newglossaryentry{STM}
{
  type=\acronymtype,
  name=STM,
  description={Short-term Memory (STM), A proposed intermediate memory system that holds information as it travels from \gls{sensory memory} to \gls{LTM} \citep[175]{anderson2005cognitive} ''},
  user1={Short-term Memory (STM)},
  first={Short-term Memory (STM)}
}
\newglossaryentry{sensory memory}
{
    name=sensory memory,
    description={Transient sensory memories hold information when it first comes in before being passed to \gls{STM}
    \citep[173]{anderson2005cognitive}},
    user1={Sensory Memory},
    plural={sensory memories}
}
\newglossaryentry{WM}
{
  type=\acronymtype,
  name=WM,
  description={Working memory (WM), is the knowledge that is currently available in memory for working on a problem. That knowledge may consist of a combination of various forms: e.g., declarative, conceptual, procedural \citep{anderson2005cognitive, Stanovich2009IntTest}},
  user1={Working Memory (WM)},
  first= {working memory (WM)}
}

\newglossaryentry{PM}
{
  type=\acronymtype,
  name=PM,
  description={Procedural Memory (PM), also referred to as \emph{procedural knowledge}. ``An implicit memory [a memory without conscious awareness] involving knowledge about how to perform tasks \citep[238]{anderson2005cognitive}'' Procedural knowledge is knowledge which we display in our behavior but of which we are not conscious \citep{anderson2005cognitive}},
  user1={Procedural Memory (PM)},
  first={Procedural Memory (PM)},
  firstplural={Procedural Memories (PMs)}
}

\newglossaryentry{production}
{
  name=production,
  description={In \gls{ACTR} declarative procedural knowledge is represented as rules called productions. \citep{ACTR_6Tutorial2012}},
  plural=productions
}

\newglossaryentry{NFL}
{
  type=\acronymtype,
  name=NFL Theorem,
  description={No Free Lunch (NFL). When solving a new engineering problem, there is no way to know, a priori, which computing or engineering approach is best or if the task is even doable, i.e., there is no free lunch \citep{Rogers2003Machinery}}
}

\newglossaryentry{cognitive decoupling}
{
  name=cognitive decoupling,
  description={``Cognitive decoupling appears to be the central cognitive operation accounting for individual differences in fluid intelligence and, because of its role in [mental] simulation and hypothetical thinking, cognitive decoupling is a crucial mental capacity. Decoupling operations help us carry out cognitive reform: the evaluation of our own beliefs and the critique of our own desires. Decoupling secondary representations from the world and then maintaining the decoupling while simulation is carried out is a Type 2 processing operation \citep{Stanovich2009IntTest}''},
  user1={Cognitive Decoupling}
}

\newglossaryentry{symbolic level}
{
  name=symbolic level,
  description={``An abstract characterization of how brain structures encode knowledge \citep{Anderson2007HowCan}''. \glslink{symbolic level}{Symbolic} and \glslink{subsymbolic level}{subsymbolic} features are designed to reflect current research and theories of human cogitative processes \citep{ACTR_6Tutorial2012}},
  user1={Symbolic Level}
}

\newglossaryentry{subsymbolic level}
{
  name=subsymbolic level,
  description={``An abstract characterization of the role of neural computation in making that [symbolic level] knowledge available \citep{Anderson2007HowCan}'' The subsymbolic level accounts for what symbols (components of \gls{DM}: episodic memory and \gls{semantic memory}) are recalled and how quickly that information is available \citep{Anderson2007HowCan}},
  user1={Subsymbolic level}
}

\newglossaryentry{cognitively plausible cohesive narrative}
{
  name=cognitively plausible cohesive narrative,
  description={A \gls{narrative} which is \gls{cognitively plausible} in the mind of the author, but not necessarily plausible in the real world, such as a science fiction \gls{narrative} or fairy tale},
  plural=cognitively plausible cohesive narratives,
  user1={Cognitively Plausible Cohesive Narrative}
}

\newglossaryentry{gustatory}
{
  name=gustatory,
  description={Concerned with taste, or the sense of taste \citep{lynott2009embodied}}
}

\newglossaryentry{bit-reduced}
{
  name=bit-reduced,
  description={Depending on the context, bit-reduced indicates the reduction in the amount of memory required to represent knowledge or the reduction in the amount of information transfer required}
}

\newglossaryentry{activation level}
{
  name=activation level,
  description={``A state of memory traces that determines both the speed and the probability of access to a memory trace
  \citep[183]{anderson2005cognitive}''},
  user1={Activation Level}
}

\newglossaryentry{meaning-based knowledge representation}
{
  name={meaning-based knowledge representation},
  description={Knowledge representations which
  are abstracts from the perceptual details and encoded into a meaning of an experience \citep[106]{anderson2005cognitive}},
  user1={Meaning-Based Knowledge Representation},
  plural=meaning-based knowledge representations
}

\newglossaryentry{perception-based knowledge representation}
{
  name=perception-based knowledge representation,
  description={``A knowledge representation that attempts to preserve much of the structure of a perceptual experience
  \citep[106]{anderson2005cognitive}''},
  user1={Perception-Based Knowledge Representation},
  plural=perception-based knowledge representations
}

\newglossaryentry{conceptual knowledge}
{
  name=conceptual knowledge,
  description={A theory of the way episodic memories are abstracted and recorded into
    \gls{LTM} is \emph{conceptual knowledge}.
    Also referred to as aggregate, abstract or blended knowledge, or \textbf{conceptual memory}.
    This knowledge is primarily \emph{declarative}, in that we are aware of we know and can usually describe to others, therefore it may also be referred to as \emph{declarative aggregate knowledge}.
    When experiences are represented in memory, every detail is not captured. Details, perceived as unimportant, are dropped from memory. Specific episodes are abstracted to general categories of experiences.  This \gls{abstraction} creates
    conceptual knowledge \citep{anderson2005cognitive}},
  user1={Conceptual Knowledge}
}

\newglossaryentry{conceptual system}
{
  name=conceptual system,
  description={ \citep{tononi2008consciousness}},
  user1={Conceptual System}
}
\newglossaryentry{aggregate memory}
{
  name=aggregate memory,
  description={see \gls{conceptual knowledge}},
  user1={Aggregate Memory},
  plural={aggregate memories}
}


\newglossaryentry{semantic memory}
{
  name=semantic memory,
  description={Semantic memories reflect general knowledge of the world, and is typically viewed as a non-instance based representation. Although we have all encountered the fact that 2+2=4 hundreds of times in our lives, we might only have one memory representation of this fact \citep[240]{anderson2005cognitive}, \citep{sims2004episodic}},
  plural=semantic memories,
  user1={Semantic Memory}
}

\newglossaryentry{semantic}
{
  name=semantic,
  description={``Grammatical rules for assigning meaning to a sentence \citep[352]{anderson2005cognitive} ''},
  plural=semantics
}

\newglossaryentry{perception}
{
  name=perception,
  description={``A way of regarding, understanding, or interpreting something; a mental impression \citep{Oxford2014} (URL Date 3 March 2014)''},
  plural=perceptions,
  user1={Perception}
}

\newglossaryentry{context}
{
  name=context,
  description={``That subset of the complete state of an individual that is used for reasoning about a given goal \citep{giunchiglia1993contextual}''},
  %description={``That subset of the complete state of an individual that is used for reasoning about a given goal \citep{giunchiglia1993contextual}'' To further clarify: \gls{context} are the elements not immediately supplied through sensor input that contribute to the reduction of uncertainty \citep{QuESTNotesApr2014}},
  plural=contexts,
  user1={Context}
}

\newglossaryentry{quale}
{
  name=quale,
  description={Plural: qualia.
  The abstracted, \gls{agent centric}, context dependent internal representations of evoked experiences based on perceived or predicted sensor data. The mental representations used throughout cognitive processing through which high-level perception, chaotic environmental stimuli are organized into the mental representations \citep{chalmers1992high, hubbard1996importance}},
  %The abstracted, \gls{agent centric}, context dependent internal representations of evoked experiences based on perceived or predicted sensor data \citep{QuESTNotesApr2014}. The mental representations used throughout cognitive processing through which high-level perception, chaotic environmental stimuli are organized into the mental representations \citep{chalmers1992high, hubbard1996importance}},
  plural=qualia,
  user1={Quale}
}
\newglossaryentry{abstraction}
{
  name=abstraction,
  description={Abstractions are a conceptualization of a set of related concepts or a recurring pattern \citep{henson2012semantic}},
  plural=abstractions,
  user1={Abstraction}
}

\newglossaryentry{perceptual modalities}
{
  name=perceptual modalities,
  description={Humans have five perceptual modalities: auditory, \gls{gustatory} (taste), haptic (touch), olfactory and visual \citep{lynott2009embodied}}
}

\newglossaryentry{CWW}
{
  type=\acronymtype,
  name=CWW,
  description={Computing with Words (CWW).  A methodology in which the objects of computation are words and propositions drawn from a natural language. Words are converted into \gls{FL} for computation, then converted back into words or phrases to provide results for human decision makeing \citep{zadeh1999computing}},
  first={Computing with Words (CWW)},
  user1={Computing with Words}
}

\newglossaryentry{FL}
{
  type=\acronymtype,
  name=FL,
  description={Fuzzy Logic (FL).  ``A method of reasoning with logical expressions describing membership in \glspl{FS}.  FL allows for degrees of truth, between 0 and 1, in a proposition, where traditional logic identifies aproposition as being true or false with a statistical degree of certainty \citep{russell2009artificial}''},
  first={Fuzzy Logic (FL)},
  user1={Fuzzy Logic}
}

\newglossaryentry{FS}
{
  type=\acronymtype,
  name=FS,
  description={Fuzzy Set (FS).  ``Fuzzy Set theory is a means of satisfying how well an object satisifies a vague description \citep{russell2009artificial}''},
  first={Fuzzy Set (FS)},
  user1={Fuzzy Set}
}

\newglossaryentry{QuEST}
{
  type=\acronymtype,
  name=QuEST,
  description={QUalia Exploitation of Sensing Technology (QuEST). Ongoing research at AFIT/AFRL addressing the limitations in computational intelligence \citep{RogersEtAl2014}},
  first={QUalia Exploitation of Sensing Technology (QuEST)},
  user1={QUalia Exploitation of Sensing Technology}
}


\newglossaryentry{CMA}
{
  type=\acronymtype,
  name=CMA,
  description={Cognitive Modeling Architecture (CMA).  A \gls{cognitive architecture} which has been implemented in a computer-based system},
  first={Cognitive Modeling Architecture (CMA)},
  firstplural={Cognitive Modeling Architectures (CMAs)},
  plural=CMAs,
  user1={Cognitive Modeling Architectures (CMAs)}
}

\newglossaryentry{IT2FLS}
{
  type=\acronymtype,
  name=IT2\textendash{}FLS,
  description={Interval Type\textendash{}II Fuzzy Logic System.  },
  first={Interval Type\textendash{}II Fuzzy Logic System (IT2\textendash{}FLS)},
  user1={Interval Type\textendash{}II Fuzzy Logic System (IT2\textendash{}FLS)}
}

\newglossaryentry{ACTR}
{
  type=\acronymtype,
  name=ACT\textendash{}R,
  description={Adaptive Control of Thought\textendash{}Rational (ACT\textendash{}R).  A modern \gls{CMA} based on the
  cognitive theory originally published by \citet{anderson1998atomic}},
  first={Adaptive Control of Thought\textendash{}Rational (ACT\textendash{}R)},
  user1={Adaptive Control of Thought\textendash{}Rational (ACT\textendash{}R)}
}


\newglossaryentry{PERC}
{
  type=\acronymtype,
  name=Per\textendash{}C,
  description={Perceptual Computer.   },
  first={Perceptual Computer (Per\textendash{}C)},
  user1={Perceptual Computer (Per\textendash{}C)}
}

\newglossaryentry{DPT}
{
  type=\acronymtype,
  name=DPT,
  description={Dual\textendash{}Process Theory of Higher Cognition (DPT). A widely accepted view that most creatures, especially humans, have two distinct cognitive processes.  One is fast and automatic, such as reactively ducking when an object comes too close to ones face.  This is referred to as \gls{Type I}, the autonomous/reactive mind. \gls{Type I} processing is often credited with intuitive behavior, such as a babies ability to suck at birth or the physical reaction to a foul odor or loud noise. The second form of cognition is when a being consciously deliberates over a decision, such as choosing a menu option, or when a squirrel learns how to defeat the expensive anti-squirrel bird feeder. This is referred to as \gls{Type II}, the deliberative/reflective mind. It is believed that these two minds work together to provide the cognitive processing for survival, problem solving as well as creativity in humans \citep{Stanovich2013DualProcess, Stanovich2009IntTest}.
},
  first={Dual\textendash{}Process Theory of Higher Cognition (DPT)},
  user1={Dual\textendash{}Process Theory of Higher Cognition (DPT)}
}

\newglossaryentry{control module}
{
  name=control module,
  description={The control module (\gls{ACTR} component) (also called the goal module) see \gls{goal module}},
  plural=control modules,
  user1={Control Module}
}

\newglossaryentry{goal module}
{
  name=goal module,
  description={The goal module (\gls{ACTR} component) (also called the control module) keeps track of one's current intentions in solving a problem \citep{Anderson2007HowCan}},
  plural=goal modules,
  user1={Goal Module}
}

\newglossaryentry{imaginal module}
{
  name=imaginal module,
  description={The imaginal module (\gls{ACTR} component) (also called the problem state module) holds the current mental representation of the  problem \citep{Anderson2007HowCan}},
  plural=imaginal modules,
  user1={Imaginal Module}
}

\newglossaryentry{problem state module}
{
  name=problem state module,
  description={The problem state module (\gls{ACTR} component) (also called the imaginal module) see \gls{imaginal module}},
  plural=problem state modules,
  user1={Problem State Module}
}

\newglossaryentry{declarative memory module}
{
  name=declarative memory module,
  description={The \gls{DM} Module (\gls{ACTR} component) is a perceptual module that perceives past experiences \citep{Anderson2007HowCan}. The currently attended to \gls{DM} is captured and stored in the \gls{retrieval buffer}},
  user1={Declarative Memory Module}
}

\newglossaryentry{retrieval buffer}
{
  name=retrieval buffer,
  description={The retrieval buffer (\gls{ACTR} component) stores the currently attended to \gls{chunk} from \gls{DM}. There is no retrieval module, per se},
  plural=retrieval buffers,
  user1={Retrieval Buffer}
}

\newglossaryentry{CryEngine}
{
  name=CryEngine\textregistered,
  description={A highly advanced development solution for the creation of blockbuster games, movies, high-quality simulations, and interactive applications \citep{CryEngine})},
  user1={CryEngine}
}

\newglossaryentry{conscious}
{
  name=conscious,
  description={See \gls{consciousness}},
  user1={Conscious}
}

\newglossaryentry{unconscious}
{
  name=unconscious,
  description={See \gls{autonomous}},
  user1={Unconscious}
}

\newglossaryentry{consciousness}
{
  name=consciousness,
  description={The experiencing of \glspl{quale} \citep{Cowell2001MindMachine}},
  user1={Consciousness}
}

\newglossaryentry{cognitively plausible}
{
  name=cognitively plausible,
  description={``To be considered cognitively plausible, a system must be capable of performing as well as humans do on cognitive tasks or be plausibly built on components that have met this test \citep{kennedycognitive}''. When modeling human performance cognitively plausibility is modeling the task the way a human does \citep{ACTR_6Tutorial2012} },
  user1={Cognitively Plausible}
}


\newglossaryentry{elementary event}
{
  name=elementary (narrative) event,
  description={The most basic, generic, entity of a \gls{narrative}. ``An elementary event describes in turn the specific behaviours (actions, processes…), experiences (situations, states…) etc., temporally and spatially constrained, that characterize some (not necessarily humans) entities or groups of entities \citep{Zarri2012ElementaryEvents}''},
  plural=elementary events,
  first=elementary (narrative) event,
  user1={Elementary Event}
}
\newglossaryentry{UTC}
{
  type=\acronymtype,
  name=UTC,
  description={Unified Theories of Cognition (UTC). The concept that an Integrated Cognitive Architecture (ICA) must incorporate all aspects of cognition. A ICA must ``\dots provide the total picture and explain the role of the parts and why they exist \citep{newell1990unified}''},
  user1={Unified Theories of Cognition},
  first={Unified Theories of Cognition (UTC)}
}

\newglossaryentry{ToM}
{
  type=\acronymtype,
  name=ToM,
  description={Theory of Mind (ToM). ``Theory of mind refers to the ability of an individual to make inferences about what others may be thinking or feeling and to predict what they may do in a given situation based on those inferences \citep{schlinger2010theory}''},
  nonumberlist,
  user1={Theory of Mind},
  first={Theory of Mind (ToM)}
}

\newglossaryentry{narrative}
{
  name=narrative,
  description={``A spoken or written account of connected events\footnote{See \gls{elementary event}}; a story. A representation of a particular situation or process in such a way as to reflect or conform to an overarching set of aims or values \citep{Oxford2014} (URL Date 9 July 2014)'' For the purposes of this research \gls{narrative} can be supplied through text format that does not represent \gls{narrative} in the traditional sense of a written story designed for a human to read. This non-traditional form of \gls{narrative} can be, for example, a text feed from a gaming system, or extracted features from a computer event log, labels, IP addresses, anything capture in symbols that represent a concept or entity},
  plural=narratives,
  user1={Narrative}
}

\newglossaryentry{agent centric}
{
  name=agent-centric,
  description={From the perspective of an individual \gls{agent}, as opposed to the term subjective which implies exclusively a human agent},
  user1={Agent-Centric}
}


\newglossaryentry{instance-based learning}
{
  name=instance-based learning,
  description={A learning theory by which each perceptual experience results in the encoding of a new instance of memory which are stored in episodic memory \citep{sims2004episodic, anderson2004integrated}},
  plural=instance-based learning,
  user1={Instance-Based Learning}
}

\newglossaryentry{commonsense knowledge}
{
  name=commonsense knowledge,
  description={``Commonsense knowledge is the collection of facts and information that an ordinary person is expected to know \citep{davis1990representations}'' A challenge for \gls{AI} and narrative technologies is the need to capture and represent commonsense knowledge which is often unstated \citep{lietohybrid} in the source data or \gls{narrative}},
  plural=commonsense knowledge,
  user1={Commonsense Knowledge}
}

\newglossaryentry{TCLI}
{
  type=\acronymtype,
  name=TCLI,
  description={Tightly Compiled Learned Information (TCLI), which is knowledge generated in \gls{WM} that has become ``\ldots{}tightly compiled and available to the autonomous mind due to overlearning and practice \citep{Stanovich2009IntTest}.'' TCLI is retained in the autonomous mind [Type I] and provided to the reflective and algorithmic minds [Type II] for production \citep{Stanovich2009IntTest}},
  user1={Tightly Compiled Learned Information (TCLI)},
  first={Tightly Compiled Learned Information (TCLI)},
  plural={Tightly Compiled Learned Information (TCLI)}
}


\newglossaryentry{ENB}
{
  type=\acronymtype,
  name=ENB,
  description={Evolutionarily compiled Encapsulated Knowledge Base (ENB).  Retained in the autonomous mind [Type I] and provided to the reflective and algorithmic minds [Type II] for production \citep{Stanovich2009IntTest} },
  user1={Encapsulated Knowledge Base (ENB)},
  first={Encapsulated Knowledge Base (ENB)},
  plural={Encapsulated Knowledge Base (ENB)}
}
\newglossaryentry{efferent}
{
  name=efferent,
  description={``Physiology: Conducted or conducting outward or away from something (for nerves, the central nervous system; for blood vessels, the organ supplied). The opposite of afferent \citep{Oxford2014}''},
  plural={efferent}
}


\newglossaryentry{eidetic}
{
  name=eidetic,
  description={``Psychology: Relating to or denoting mental images having unusual vividness and detail, as if actually visible \citep{Oxford2014}''},
}

\newglossaryentry{concept-base}
{
  name=concept\textendash{}base,
  description={a repository of  \glslink{symbolic level}{symbolic} elements representing \gls{conceptual knowledge} and \gls{context}}
}

\newglossaryentry{CLARION}
{
  type=\acronymtype,
  name=CLARION,
  description={Connectionist Learning with Adaptive Rule Induction On-line (CLARION)},
  user1={Connectionist Learning with Adaptive Rule Induction On-line (CLARION)},
  first={Connectionist Learning with Adaptive Rule Induction On-line (CLARION)}
}

\newglossaryentry{ACS}
{
  type=\acronymtype,
  name=ACS,
  description={action-centered subsystem (ACS). (\gls{CLARION} component. (procedural)},
  user1={Action-Centered Subsystem (ACS)},
  first={action-centered subsystem (ACS)},
  plural={ACSs}
}

\newglossaryentry{NACS}
{
  type=\acronymtype,
  name=NACS,
  description={non-action-centered subsystem (NACS). (\gls{CLARION} component. (declarative)},
  user1={Non-Action-Centered Subsystem (NACS)},
  first={non-action-centered subsystem (NACS)},
  plural={NACSs}
}


\newglossaryentry{IDNs}
{
  type=\acronymtype,
  name=IDNs,
  description={Implicit Decision Networks (IDNs). (\gls{CLARION} component},
  user1={Implicit Decision Networks (IDNs)},
  first={Implicit Decision Networks (IDNs)},
  plural={IDNs}
}

\newglossaryentry{ARS}
{
  type=\acronymtype,
  name=ARS,
  description={Action Rule Store (ARS). (\gls{CLARION} component},
  user1={Action Rule Store (ARS)},
  first={Action Rule Store (ARS)},
  plural={ARSs}
}

\newglossaryentry{MS}
{
  type=\acronymtype,
  name=MS,
  description={Motivational Subsystem (MS). (\gls{CLARION} component},
  user1={Motivational Subsystem (MS)},
  first={Motivational Subsystem (MS)},
  plural={MSs}
}

\newglossaryentry{MCS}
{
  type=\acronymtype,
  name=MS,
  description={Meta-Cognitive Subsystem (MCS). (\gls{CLARION} component},
  user1={Meta-Cognitive Subsystem (MCS)},
  first={Meta-Cognitive Subsystem (MCS)},
  plural={MCSs}
}

\newglossaryentry{SA}
{
  type=\acronymtype,
  name=SA,
  description={Situation Awareness (SA). },
  user1={Situation Awareness (SA)},
  first={Situation Awareness (SA)}
}

\newglossaryentry{prototypical schema}
{
  name=prototypical schema,
  description={},
  plural={prototypical schemas},
  user1={Prototypical Schema}
}


\newglossaryentry{simplicial complex}
{
  name=simplicial complex,
  description={``a set of simplices with all of their faces \citep{johnson2013hypernetworks}''},
  user1={Simplicial Complex},
  plural={simplicial complexes}
}
\newglossaryentry{simplicial family}
{
  name=simplicial family,
  description={``a simplicial complex where all the faces of all the simplices are not present, i.e., any set of simplices forms a simplicial family \citep{johnson2013hypernetworks}''},
  plural={simplicial families}
}

\newglossaryentry{simplex}
{
  name=simplex,
  description={``A simplex is the generalization of a tetrahedral [or greater \citep{johnson1995multidimensional}] region of space to $n$ dimensions.  The boundary of a $k$-simplex has $k+1$ $0$-faces (vertices),  $k(k+1)/2$ $1$-faces (edges) and $\binom{k+1}{i+1}$ $i$-faces, where $\binom{n}{k}$ is a binomial coefficient \citep{WolframMathWorld}''},
  plural={simplices}
}

\newglossaryentry{abstract p-simplex}
{
  name=abstract p-simplex,
  description={``Let $V$ be a set whose element are called vertices. Any subset of $V$,$%
\{v_{0},v_{1},\ldots ,v_{p}\}$ determines an object called an abstract
p-simplex, written $\sigma =\langle v_{0},v_{1},\ldots ,v_{p}\rangle $. A
p-simplex can be represented by a p-dimensional polyhedhron in$(p+k)$%
-dimensional space, where $k\geq 0$ \citep{johnson2013hypernetworks}''}
}


\newglossaryentry{polyhedral dynamics}
{
  name=polyhedral dynamics,
  description={Polyhedral dynamics (aka Hypernetwork, aka Q-Analysis) developed from set theory, and is entirely compatible with
network theory, as a methodology to address the analysis of large-scale
systems represented in multidimensional arrays, the relationship between qualitative data, psychological and social relations \citep{casti1977polyhedral, johnson2013hypernetworks, PolyhedralEmpowerment}}
}


\newglossaryentry{polyhedron}
{
  name=polyhedron,
  description={``The word polyhedron has slightly different meanings in geometry and algebraic [topology]. In geometry, a polyhedron is simply a three-dimensional solid which consists of a collection of polygons, usually joined at their edges. \citep{WolframMathWorld}''
  ``In algebraic topology it is defined as a space that can be built from such \emph{building blocks} as line segments, triangles, tetrahedra, and their higher dimensional analogs by \emph{gluing them together} along their faces (Munkres 1993).  More specifically, it can be defined as the underlying space of a simplicial complex (with the additional constraint sometimes imposed that the complex be finite; Munkres 1993). \citep{monk1974connections, WolframMathWorld}''},
  plural={polyhedra}
}
\newglossaryentry{incidence matrix}
{
  name=incidence matrix,
  description={``The incidence matrix of a graph gives the $(0,1)$-matrix which has a row for each vertex and column for each edge, and $(v,e)=1$ iff vertex $v$ is incident upon edge $e$ \citep{WolframMathWorld}''},
  plural={incidence matrices}
}

\newglossaryentry{backcloth}
{
  name=backcloth,
  description={``the relatively unchanged network or simplicial complex structure \citep{johnson2013hypernetworks}''}
}
\newglossaryentry{traffic}
{
  name=traffic,
  description={``the activity (numbers) on the backcloth \citep{johnson2013hypernetworks}''}
}

\newglossaryentry{hypersimplex}
{
  name=hypersimplex,
  description={``a simplex augmented by it's relation. Hypersimplices are structured sets of vertices and can discriminate different relations on the same set of parts \citep{johnson2013hypernetworks}''},
  plural={hypersimplices}
}

\newglossaryentry{hypergraph}
{
  name=hypergraph,
  description={``Provide a method of representing relationships between more than two things. a set of vertices, V, and a set of subsets of V, E called hypergraph edges. In general the members of E can have more than two elements \citep{johnson2013hypernetworks}''}
}

\newglossaryentry{bipartite relation}
{
  name=bipartite relation,
  description={``a relation between two sets $A$ and $B$ where $A\cap B=\varnothing $ \citep{johnson2013hypernetworks}''}
}


\newglossaryentry{relation}
{
  name=relation,
  description={``A relation, $R$, between two sets $A$ and $B$ is a rule that establishes for each $a$ in $A$ and each $b$ in $B$ whether or not a is $R$-related to $b$. \citep{johnson2013hypernetworks}''},
  plural={relations}
}


\newglossaryentry{intersection of two sets}
{
  name=intersection of two sets,
  description={``The intersection of two sets $A$ and $B$, written as $A\cap B$ is the set of elements belonging to them both, $A\cap B=\{x|x\in A$ and $x\in B\}$  \citep{johnson2013hypernetworks}''}
}

\newglossaryentry{intersection of two simplices}
{
  name=intersection of two simplices,
  description={``The intersection of two simplices $\sigma $\ and $\sigma ^{\prime }$\ is
defined to be their highest dimensional shared face, $\sigma ^{\prime \prime
}$. We write $\sigma \cap \sigma ^{\prime }=\sigma ^{\prime \prime }$ \citep{johnson2013hypernetworks}''}
}

\newglossaryentry{union}
{
  name=union,
  description={``The union of two sets $A$ and $B$, written as $A\cup B$ is the set of
elements belonging to $A$, to $B$, or both, $A\cup B=\{x|x\in A$ or $x\in B\}
$ \citep{johnson2013hypernetworks}''}
}


\newglossaryentry{symmetric difference}
{
  name=symmetric difference,
  description={``The symmetric difference (also called exclusive 'or') of sets A and B, is
  the set of elements that belong to A or belong to B but do not belong
  to both. The symmetric difference is written $A\Delta B$ $\overset{def}{=}$ $\{x\vert x\in A$ or $x\in B$ but $x\notin A\cap B\}$
\citep{johnson2013hypernetworks}''}
}

\newglossaryentry{difference}
{
  name=difference,
  description={``The difference of two sets $A$ and $B$, written as $A\backslash B$ is the
set of elements belonging to $A$, but not belonging to $B$,  $A\backslash
B=\{x|x\in A$ and $x\notin B\}$ \citep{johnson2013hypernetworks}''}
}

\newglossaryentry{class}
{
  name=class,
  description={``A class is a set whose members are other sets \citep{johnson2013hypernetworks}''}
}

\newglossaryentry{power set}
{
  name=power set,
  description={``The power set of a set $A$ is the set of all subsets of $A$. The power set is denoted by $\rho (A)$ or $2^{A}$ with $\rho (A)=\{A^{\prime }|$ for all $A^{\prime }\subseteq A\}
$ \citep{johnson2013hypernetworks}''}
}

\newglossaryentry{mapping}
{
  name=mapping,
  description={``Let $A$ and $B$ bet sets. A mapping is rule, $f$ , that assigns an element
$f(A)\in B$ to each $a\in A$. $A$ is called the domain of the mapping and $B$
is called the codomain or range \citep{johnson2013hypernetworks}''}
}

\newglossaryentry{Galois pair}
{
  name=Galois pair,
  description={``In a Galois pair $A^{\prime }\leftrightarrow B^{\prime }$ every $a$ in $A^{\prime }$ is
R-related to every $b$ in $B^{\prime }$ \citep{johnson2013hypernetworks}''}
}

\newglossaryentry{Galois prism}
{
  name=Galois prism,
  description={``The Galois prism of a pair $A^{\prime }\leftrightarrow B^{\prime }$ is defined to be their prism $\sigma \diamond \sigma ^{\prime }$ \citep{johnson2013hypernetworks}''}
}


\newglossaryentry{eccentricity}
{
  name=eccentricity,
  description={Hypernetwork Theory: Eccentricity is an asymmetric measure of connectivity between two simplices \citep{johnson2013hypernetworks}},
  plural={eccentricities}
}

\newglossaryentry{simplicial prism}
{
  name=simplicial prism,
  description={``The prism $\sigma \diamond \sigma ^{\prime }$  between $\sigma $ and $\sigma
^{\prime }$, is the simplex with the property that $\left\langle
x\right\rangle \overset{<}{\sim}\sigma \diamond \sigma ^{\prime }$ if
and only if $\left\langle x\right\rangle \overset{<}{\sim}\sigma $ or $%
\left\langle x\right\rangle \overset{<}{\sim}\sigma $\  $\left\langle
x\right\rangle \overset{<}{\sim}\sigma ^{\prime }$ or both \citep{johnson2013hypernetworks}''}
}

\newglossaryentry{difference prism}
{
  name=difference prism,
  description={``The difference prism, $\sigma \Delta \sigma ^{\prime }$ of two simplices $%
\sigma $ and $\sigma ^{\prime }$, is defined to be the prism between thier
eccentric faces, $\sigma \Delta \sigma ^{\prime }=(\sigma \smallsetminus
\sigma ^{\prime })\diamond (\sigma ^{\prime }\smallsetminus \sigma )$ \citep{johnson2013hypernetworks}''}
}

\newglossaryentry{Q-analysis}
{
  name=Q-analysis,
  description={``The mathematical formalism which is the basis of Hypernetwork Theory. The first paper on Q-analysis was published by Atkin et al in 1971 \citep{johnson2013hypernetworks}''}
}
\newglossaryentry{hypernetwork theory}
{
  name=hypernetwork theory,
  description={aka Hypernetworks, a theory which extends network theory to multidimensional hypernetworks for modeling multi-element relationships, in particular systems in nature, society and cognition. Also referred to as \Gls{polyhedral dynamics} in earlier literature \citep{casti1977polyhedral, johnson2013hypernetworks, wang2010evolving}},
  user1={Hypernetwork Theory}
}


\newglossaryentry{hypernetwork}
{
  name=hypernetwork,
  description={``a set of \gls{hypersimplices} \citep{johnson2013hypernetworks}''},
  plural=hypernetworks
}

\newglossaryentry{hypersimplices}
{
  name=hypersimplices,
  description={``structured sets of vertices which exist at a higher level of representation than its vertices \citep{johnson2013hypernetworks}''}
}

\newglossaryentry{PCT}
{
  type=\acronymtype,
  name=PCT,
  description={Parsimonious Covering Theory (PCT), primarily used for medical diagnosis,
  PCT uses domain-specific background knowledge to determine
  the best explanation (e.g., diagnosis) for a set of observations \citep{henson2012semantic}},
  user1={Parsimonious Covering Theory (PCT)},
  first= {Parsimonious Covering Theory (PCT)}
}

\newglossaryentry{schemata}
{
  name=schemata,
  description={\textbf{(aka Schematic memory)} ``Existing knowledge providing a framework within which new knowledge is integrated \citep{YoungMichael1998}.'' F.C. Bartlett's work \citep{bartlett1932remembering} (as cited in M. J. Young, 1998) ``An active organization of past experiences and reactions that shapes a person's response to new stimuli \citep{YoungMichael1998}.'' J. M. Mandler's work (as cited in M. J. Young, 1998) ``A modern definition of the schema concept defines a schema as a temporally or spatially organized structure whose components are a set of variables or slots that are filled or instantiated by values.  Mandler proposes that schema are formed on the basis of proximities experienced in space or time \citep{YoungMichael1998}''},
  plural={schemata},
  user1={Schemata}
}
\newglossaryentry{cognitive science}
{
  name=Cognitive Science,
  description={``The interdisciplinary field which brings together computer models from \gls{AI} and experimental techniques from psychology to construct precise and testable theories of the human mind \citep{russell2009artificial}''}
}
\newglossaryentry{DLL}
{
  type=\acronymtype,
  name=DLL,
  description={Dynamic-Link Library (DLL)},
  first={Dynamic-Link Library (DLL)},
  user1={Dynamic-Link Library (DLL)}
}
\newglossaryentry{OS}
{
  type=\acronymtype,
  name=OS,
  description={Operating System (OS)},
  first={Operating System (OS)}
}
\newglossaryentry{SMTP}
{
  type=\acronymtype,
  name=SMTP,
  description={Simple Mail Transfer Protocol (SMTP)},
  first={Simple Mail Transfer Protocol (SMTP)}
}
\newglossaryentry{API}
{
  type=\acronymtype,
  name=API,
  description={Application Programmer Interface (API)},
  first={Application Programmer Interface (API)},
  user1={Application Programmer Interface (API)}
}

\newglossaryentry{AI}
{
  type=\acronymtype,
  name=AI,
  description={Artificial Intelligence (AI). The field of AI incorporates four broad areas of computational sciences: thinking humanly, thinking rationally, acting humanly and acting rationally \citep{russell2009artificial}},
  first={Artificial Intelligence (AI)}
}

\newglossaryentry{CI}
{
  type=\acronymtype,
  name=CI,
  description={Computational Intelligence (CI). ``The study and design of intelligent agents \citep{poole1998computational} (as cited \citep{russell2009artificial})''},
  first={Computational Intelligence (CI)}
}
\newglossaryentry{PE}
{
  type=\acronymtype,
  name=PE,
  description={Portable Executable (PE)},
  first={Portable Executable (PE)}
}
\newglossaryentry{IIT}
{
  type=\acronymtype,
  name=IIT,
  description={The Integrated Information Theory (IIT) of Consciousness equates consciousness with integrated information \citep{tononi2012integrated}},
  first={Integrated Information Theory (IIT) of Consciousness},
  user1={Integrated Information Theory (IIT) of Consciousness}
}
\newglossaryentry{AV}
{
  type=\acronymtype,
  name=AV,
  description={Antivirus (AV) software},
  first={Antivirus (AV) software},
  user1={Antivirus (AV)}
}
\newglossaryentry{AVIRA}
{
  type=\acronymtype,
  name=AVIRA,
  description={AVIRA (Awareness, Vision, Imagination, Responsibility, Action) Antivirus vendor},
  first={AVIRA}
}
\newglossaryentry{VirusTotal}
{
  name=VirusTotal,
  description={VirusTotal},
  first={VirusTotal}
}
\newglossaryentry{Kaspersky}
{
  name=Kaspersky,
  description={Kaspersky Antivirus vendor},
  first={Kaspersky}
}
\newglossaryentry{Symantec}
{
  name=Symantec,
  description={Symantec Antivirus vendor},
  first={Symantec}
}
\newglossaryentry{type-classification}
{
  name=type-classification,
  description={The classification of potentially malicious software (malware) into categories based on various attributes, such as, spreading mechanism, destructive behavior, system dependencies, specific \glspl{tv}, etc\dots \citep{szor2005art}},
  first={(malware) type-classification}
}
\newglossaryentry{CME}
{
  type=\acronymtype,
  name=CME,
  description={Common Malware Enumeration (CME) initiative by Mitre Corporation},
  first={Common Malware Enumeration (CME) initiative by Mitre}
}
\newglossaryentry{CARO}
{
  type=\acronymtype,
  name=CARO,
  description={Computer \-Anti-Virus \-Researchers' \-Organization (CARO)},
  first={Computer \-Anti-Virus \-Researchers' \-Organization (CARO)}
}
\newglossaryentry{packer}
{
  name=packer,
  description={Packers are wrappers put around pieces of software to compress and/or encrypt their contents \citep{virusbtnpacker}.},
  first={packer}
}
\newglossaryentry{sigcheck}
{
  name=sigcheck,
  description={Sigcheck is a command-line utility that shows file version number, timestamp information, and digital signature details, including certificate chains \citep{SigCheck}.},
  first={sigcheck}
}
\newglossaryentry{cuckoo}
{
  name=cuckoo,
  description={Cuckoo Sandbox is a malware analysis system \citep{cuckooSandBox}},
  first={Cuckoo Sandbox}
}
\newglossaryentry{Mutex}
{
  type=\acronymtype,
  name=Mutex,
  description={Mutex, Short for mutual exclusion object. A Mutex is a program object that allows multiple program threads to share the same resource, such as file access, but not simultaneously. \citep{webopedia} (/term/m/mutex.html))},
  first={Mutex}
}
\newglossaryentry{JSON}
{
  type=\acronymtype,
  name=JSON,
  description={JSON, (JavaScript Object Notation) },
  first={JavaScript Object Notation (JSON)}
}

\newglossaryentry{metric space}
{
  name=metric space,
  description={A precise formal measure of distance given any set. A abstracted notion of Euclidean spaces applied to any set \citep{peeler2011metric}}
}

\newglossaryentry{metric}
{
  name=metric,
  description={``A formal way to view the notion of distance between [elements] in a set. \citep{peeler2011metric}''}
}

\newglossaryentry{trojan}
{
  name=trojan,
  description={A trojan (horse) is a malicious program that masquerades as a benign or useful program. Trojans, unlike viruses, do not self-replicate. Their method of infection is manual and may include \gls{IRC}, \gls{P2P}, e-mails, file transfers, etc. \citep{mcafee}}
}

\newglossaryentry{IRC}
{
  type=\acronymtype,
  name=IRC,
  description={IRC, (Internet Relay Chat) },
  first={Internet Relay Chat (IRC)}
}

\newglossaryentry{P2P}
{
  type=\acronymtype,
  name=P2P,
  description={P2P, (Peer-to-peer) },
  first={Peer-to-peer (P2P)}
}

\newglossaryentry{RPC}
{
  type=\acronymtype,
  name=RPC,
  description={Remote Procedure Call (RPC)},
  first={Remote Procedure Call (RPC)}
}
\newglossaryentry{LSA}
{
  type=\acronymtype,
  name=LSA,
  description={Local Security Authority (LSA)},
  first={Local Security Authority (LSA)}
}

\newglossaryentry{kNN}
{
  type=\acronymtype,
  name=kNN,
  description={k Nearest Neighbors},
  first={k-nearest neighbors (kNN)},
  user1={K-Nearest Neighbors (kNN)}
}
\newglossaryentry{DT}
{
  type=\acronymtype,
  name=DT,
  description={decision tree (DT)},
  first={decision tree (DT)},
  user1={Decision Tree (DT)}
}
\newglossaryentry{QS}
{
  type=\acronymtype,
  name=QS,
  description={Qualia Space (QS). A component of  the \gls{IIT} which equates consciousness with integrated information.
  ``Qualia space (QS) is a space where each axis represents a possible state of the complex, each point is a
  probability distribution of its states, and arrows between points represent the informational relationships
  among its elements generated by causal mechanisms (connections). Together, the set of informational relationships
  within a complex constitute a shape in QS that completely and univocally specifies a particular experience \citep{tononi2008consciousness}''},
  first={qualia space (QS)},
  user1={Qualia Space (QS)}
}
\newglossaryentry{GWT}
{
  type=\acronymtype,
  name=GWT,
  description={Global Workspace Theory (GWT)},
  first={Global Workspace Theory (GWT)}
}
\newglossaryentry{QMA}
{
  type=\acronymtype,
  name=QMA,
  description={Qualia Modeling Agent (QMA)},
  first={Qualia Modeling Agent (QMA)},
  user1={Qualia Modeling Agent (QMA)}
}
\newglossaryentry{BICA}
{
  type=\acronymtype,
  name=BICA,
  description={Biologically Inspired Cognitive Architectures (BICA)},
  first={Biologically Inspired Cognitive Architectures (BICA)},
  firstplural={Biologically Inspired Cognitive Architectures (BICAs)},
  user1={Biologically Inspired Cognitive Architectures (BICA)}
}

\newglossaryentry{OOM}
{
  type=\acronymtype,
  name=OOM,
  description={order of magnitude (OOM)},
  first={Order Of Magnitude (OOM)},
  firstplural={Order Of Magnitudes (OOMs)},
  user1={Order Of Magnitude (OOM)}
}
\newglossaryentry{ML}
{
  type=\acronymtype,
  name=ML,
  description={Machine Learning (ML)},
  first={machine learning (ML)},
  user1={Machine Learning (ML)}
}

\newglossaryentry{Sys1}
{
  type=\acronymtype,
  name=Sys1,
  description={\emph{System 1} (Sys1), see \gls{Type I}},
  first={\emph{System 1} (Sys1)}
}
\newglossaryentry{Sys2}
{
  type=\acronymtype,
  name=Sys2,
  description={\emph{System 2} (Sys2), see \gls{Type II}},
  first={\emph{System 2} (Sys2)}
}
\newglossaryentry{ANOVA}
{
  type=\acronymtype,
  name=ANOVA,
  description={analysis of variance (ANOVA)},
  first={analysis of variance (ANOVA)}
}
\newglossaryentry{OFC}
{
  type=\acronymtype,
  name=OFC,
  description={orbitofrontal cortex (OFC)},
  first={orbitofrontal cortex (OFC)},
  firstplural={orbitofrontal cortices (OFCs)}
}
\newglossaryentry{CS}
{
  type=\acronymtype,
  name=CS,
  description={conditioned stimuli (CS)},
  first={conditioned stimuli (CS)}
}
\newglossaryentry{US}
{
  type=\acronymtype,
  name=US,
  description={unconditioned stimuli (US)},
  first={unconditioned stimuli (US)}
}
\newglossaryentry{CERA-CRANIUM}
{
  type=\acronymtype,
  name=CERA-CRANIUM,
  description={Conscious and Emotional Reasoning Architecture--Cognitive Robotics Architecture Neurologically Inspired Underlying Manager (CERA-CRANIUM)},
  first={Conscious and Emotional Reasoning Architecture--Cognitive Robotics Architecture Neurologically Inspired Underlying Manager (CERA-CRANIUM)}
}
\newglossaryentry{LIDA}
{
  type=\acronymtype,
  name=LIDA,
  description={Learning Intelligent Distribution Agent (LIDA)},
  first={Learning Intelligent Distribution Agent (LIDA)}
}
\newglossaryentry{CASE}
{
  type=\acronymtype,
  name=CASE,
  description={Conscious Architecture for State Exploitation (CASE)},
  first={Conscious Architecture for State Exploitation (CASE)}
}
\newglossaryentry{MC}
{
  type=\acronymtype,
  name=MC,
  description={Machine Conscious (MC)},
  first={Machine Conscious (MC)}
}
\newglossaryentry{ANN}
{
  type=\acronymtype,
  name=ANN,
  description={artificial neural networks (ANNs)},
  firstplural={artificial neural networks (ANNs)},
  first={artificial neural network (ANN)}
}
\newglossaryentry{TCP/IP}
{
  type=\acronymtype,
  name=TCP/IP,
  description={Transmission Control Protocol/Internet Protocol (TCP/IP)},
  first={Transmission Control Protocol/Internet Protocol (TCP/IP)}
}
\newglossaryentry{TCP}
{
  type=\acronymtype,
  name=TCP,
  description={Transmission Control Protocol (TCP)},
  first={Transmission Control Protocol (TCP)}
}
\newglossaryentry{GUI}
{
  type=\acronymtype,
  name=GUI,
  description={Graphical User Interface (GUI)},
  first={Graphical User Interface (GUI)}
}
\newglossaryentry{OLE}
{
  type=\acronymtype,
  name=OLE,
  description={Object Linking and Embedding (OLE)},
  first={Object Linking and Embedding (OLE)}
}
\newglossaryentry{SENS}
{
  type=\acronymtype,
  name=SENS,
  description={System Event Notification Service (SENS)},
  first={System Event Notification Service (SENS)}
}
\newglossaryentry{SSL}
{
  type=\acronymtype,
  name=SSL,
  description={Secure Socket Layer (SSL)},
  first={Secure Socket Layer (SSL)}
}
\newglossaryentry{UNC}
{
  type=\acronymtype,
  name=UNC,
  description={Universal (or Uniform, or Unified)[file] Naming Convention (UNC)},
  first={Universal Naming Convention (UNC)}
}
\newglossaryentry{URL}
{
  type=\acronymtype,
  name=URL,
  description={Uniform Resource Locator (URL)},
  first={Uniform Resource Locator (URL)}
}
\newglossaryentry{ASCII}
{
  type=\acronymtype,
  name=ASCII,
  description={American Standard Code for Information Interchange (ASCII)},
  first={ASCII}
}
\newglossaryentry{RRAS}
{
  type=\acronymtype,
  name=RRAS,
  description={Routing and Remote Access Service (RRAS)},
  first={Routing and Remote Access Service (RRAS)}
}

\newglossaryentry{VPN}
{
  type=\acronymtype,
  name=VPN,
  description={Virtual Private Network (VPN)},
  first={Virtual Private Network (VPN)}
}


\newglossaryentry{DNS}
{
  type=\acronymtype,
  name=DNS,
  description={Domain Name System (Server) (DNS)},
  first={Domain Name System (DNS)}
}
\newglossaryentry{OTU}
{
  type=\acronymtype,
  name=OTU,
  plural={OTUs},
  description={Operational Taxonomic Unit (OTU)},
  firstplural={Operational Taxonomic Units (OTUs)},
  first={Operational Taxonomic Unit (OTU)}
}
\newglossaryentry{STF}
{
  name=Stanovich's framework,
  description={An explanation for human behavior by describing the interaction of the three minds or cognitive levels \citep{Stanovich2009IntTest}},
  first={Stanovich's tripartite framework},
  user1={Stanovich's Tripartite Framework}
}
\newglossaryentry{CL}
{
  type=\acronymtype,
  name=CL,
  description={Common Lisp (CL) programming language},
  first={Common Lisp (CL)}
}
\newglossaryentry{UCI}
{
  type=\acronymtype,
  name=UCI,
  description={University of California, Irvine (UCI)},
  first={University of California, Irvine (UCI)}
}
\newglossaryentry{malware}
{
  type=\acronymtype,
  name=malware,
  description={malicious software (malware)},
  first={malicious software (malware)}
}
\newglossaryentry{URI}
{
  type=\acronymtype,
  name=URI,
  description={Uniform Resource Identifier (URI)},
  first={Uniform Resource Identifier (URI)}
}
\newglossaryentry{HTTP}
{
  type=\acronymtype,
  name=HTTP,
  description={Hypertext Transfer Protocol (HTTP)},
  first={Hypertext Transfer Protocol (HTTP)}
}
\newglossaryentry{SME}
{
  type=\acronymtype,
  name=SME,
  description={subject matter expert (SME)},
  first={subject matter expert (SME)},
  firstplural={subject matter experts (SMEs)}
}
\newglossaryentry{EPIC}
{
  type=\acronymtype,
  name=EPIC,
  description={Executive Process-Interactive Control (EPIC)},
  first={Executive Process-Interactive Control (EPIC)}
}
\newglossaryentry{ICARUS}
{
  type=\acronymtype,
  name=ICArUS,
  description={Integrated Cognitive\textendash{}neu\-ro\-science Architectures for Understanding Sensemaking (ICArUS)},
  first={Integrated Cognitive\textendash{}neu\-ro\-science Architectures for Understanding Sensemaking (ICArUS)}
}

\newglossaryentry{ADAPT}
{
  type=\acronymtype,
  name=ADAPT,
  description={Adaptive Dynamics and Active Perception for Thought (ADAPT), a symbolic cognitive architecture \citep{benjamin2004adapt}},
  first={Adaptive Dynamics and Active Perception for Thought (ADAPT)}
}

\newglossaryentry{SASE}
{
  type=\acronymtype,
  name=SASE,
  description={Self-Aware Self-Effecting (SASE)},
  first={Self-Aware Self-Effecting (SASE)}
}

\newglossaryentry{CMN}
{
  type=\acronymtype,
  name=CMN,
  description={Computational Models of Narrative (CMN)},
  first={Computational Models of Narrative (CMN)},
  user1={Computational Models of Narrative (CMN)}
}

\newglossaryentry{fMRI}
{
  type=\acronymtype,
  name=fMRI,
  description={functional Magnetic Resonance Imaging (fMRI)},
  first={functional Magnetic Resonance Imaging (fMRI)}
}
\newglossaryentry{pdf}
{
  type=\acronymtype,
  name=pdf,
  description={Probability density function (pdf)},
  plural={pdfs},
  first={probability density function (pdf)},
  firstplural={probability density functions (pdfs)}
}
\newglossaryentry{mpd}
{
  type=\acronymtype,
  name=mpd,
  description={marginal probability distribution (mpd)},
  first={marginal probability distribution (mpd)}
}
\newglossaryentry{ARPA}
{
  type=\acronymtype,
  name=ARPA,
  description={Advanced Research Projects Activity (ARPA)},
  first={Advanced Research Projects Activity (ARPA)}
}
\newglossaryentry{Marginal probability}
{
  name=Marginal probability,
  description={Marginal probability: the probability of an event occurring (p(A)), it may be thought of as an unconditional probability.  It is not conditioned on another event.  Example:  the probability that a card drawn is red (p(red) = 0.5).  Another example:  the probability that a card drawn is a 4 (p(four)=1/13)},
  %See more at: http://sites.nicholas.duke.edu/statsreview/probability/jmc/#sthash.qIAQxAJp.dpuf
  first={Marginal probability}
}
\newglossaryentry{TL}
{
  name=TL,
  type=\acronymtype,
  description={Transfer Learning (TL) In the research field of \gls{ML}, transfer learning is the ability of a system to apply knowledge or skills learned in previous tasks to subsequent tasks or new domains, which are similar in some way \citep{pan2010survey}},
  first={Transfer Learning (TL)},
  user1={Transfer Learning (TL)}
}

\newglossaryentry{Hausdorff distance}
{
  name=Hausdorff distance,
  description={``Given two sets of points, \dots the maximum of the distance from a point in any of the sets to the nearest point in the other set \citep{rote1991computing}''}
}
\newglossaryentry{cd}
{
  name=concept drift,
  description={Concept drift refers to a learning problem that changes over time. In particular, the statistical properties of the \gls{tv}, which the model is trying to predict, change over time in unforeseen ways \citep{vzliobaite2010learning}},
  user1={Concept Drift}
}
\newglossaryentry{supervised machine learning}
{
  name=supervised machine learning,
  description={``Supervised machine learning is the search for algorithms that reason from externally supplied instances
to produce general hypotheses, which then make predictions about future instances. In other words, the
goal of supervised learning is to build a concise model of the distribution of class labels in terms of
predictor features. The resulting classifier is then used to assign class labels to the testing instances
where the values of the predictor features are known, but the value of the class label is unknown \citep{kotsiantis2007supervised}''}
}
\newglossaryentry{fluid intelligence}
{
  name=fluid intelligence,
  description={Intelligence-as-process, is the ability to reason over a variety of domains and is measured by abstract reasoning, analogous reasoning and series completion. Fluid intelligence is generally a \gls{Type II} process \citep{Stanovich2009IntTest}}
}

\newglossaryentry{crystalline intelligence}
{
  name=crystalline intelligence,
  description={Intelligence-as-knowledge, is declarative knowledge gained through experiences, and is measured by vocabulary and general knowledge tests. Crystalline intelligence is generally a \gls{Type I} process \citep{Stanovich2009IntTest}}
}
\newglossaryentry{preattentive process}
{
  name=preattentive process,
  description={The process by which \gls{Type II} receives input from \gls{Type I} and establish immediate goals \citep{Stanovich2009IntTest}},
  plural={preattentive processes}
}
\newglossaryentry{Query Simplex}
{
  name=Query Simplex,
  description={In QMA: A simplex with one \gls{Query Element}},
  %description={In QMA: A simplex with one \gls{Query Element} \citep{vaughan2016BICA}},
  user1={Query Simplex}
}
\newglossaryentry{tv}
{
  name=target variable,
  description={In QS: The variable, which is the target of the inference/pattern-completion formalism, whose value is to be inferred. The term \emph{target variable} is also referred to as \emph{class} or \emph{category} in \gls{ML} terminology \citep{pang2006introduction}},
  user1={Target Variable}
}
\newglossaryentry{Query Element}
{
  name=Query Element,
  description={In \gls{QMA}: The Query Element is the specific element that is to be inferred. An observation/experience can have multiple unobserved elements, but only one (at a time) can be a Query Element.},
  user1={Query Element}
}

\newglossaryentry{ISR}
{
  name=ISR,
  type=\acronymtype,
  description={Intelligence, Surveillance, and Reconnaissance (ISR)},
  first={Intelligence, Surveillance, and Reconnaissance (ISR)}
}

\newglossaryentry{FDR}
{
  name=FDR,
  type=\acronymtype,
  description={fully disjunctive reasoning (FDR)},
  first={fully disjunctive reasoning (FDR)}
} 